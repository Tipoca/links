\documentclass[11pt,a4paper]{article}

\title{Links Grammar}
\date{}

\usepackage{shortvrb}
\usepackage{longtable}
\begin{document}
\maketitle
\tableofcontents

\begingroup
\catcode`<=\active
\catcode`>=\active
\catcode`!=\active
\gdef\setupgrammardefs{
  \catcode`\<=\active \def <{$\langle$}
  \catcode`\>=\active \def >{$\rangle$}
  \catcode`\!=\active \def !{$\mid$}
}
\endgroup

\newenvironment{grammar}{
  \setupgrammardefs
  \begin{itshape}
    \begin{small}
      \MakeShortVerb{\'}
\begin{longtable}{llll}
}{
\end{longtable}
    \DeleteShortVerb{\'}
    \end{small}
  \end{itshape}
}

\section{Grammar for patterns}

\begin{grammar}
pattern &::=& as-pattern < ':' primary-datatype >  \\
\\
as-pattern &::=& cons-pattern < 'as' 'VARIABLE' >  \\
\\
cons-pattern &::=& constructor-pattern < '::' cons-pattern > \\
\\
constructor-pattern &::=& primary-pattern \\
&&                        'CONSTRUCTOR' < parenthesized-pattern > \\
\\
parenthesized-pattern &::=& '(' < pattern < ',' patterns >> ) \\
&&                         '(' labeled-patterns < ! pattern > ) \\
\\
primary-pattern &::=& 'VARIABLE' \\
&&                     '-' \\
&&                     special-constant \\
&&                     '[' < patterns > ']' \\
&&                     parenthesized-pattern \\
\\
patterns &::=& pattern < ',' patterns > \\
\\
labeled-patterns &::=& record-label = pattern < , labeled-patterns > \\
\end{grammar}

\section{Grammar for types}

\begin{grammar}
datatype &::=& mu-datatype < '->' datatype > \\
\\
mu-datatype &::=& 'mu' 'VARIABLE' '.' mu-datatype \\
&&                primary-datatype \\
\\
primary-datatype &::=& '(' datatype ')' \\
&&                     '(' < datatype , datatypes > ')' \\
&&                     '(' fields ')' \\
&&                     '{' 'VARIABLE' '}' \\
&&                     TableHandle '(' zfields ')' \\
&&                     '[[' < vfields > ']]' \\
&&                     '[' datatype ']' \\
&&                     'VARIABLE' \\
&&                     'CONSTRUCTOR' '('< primary-datatype-list >')' \\
\\
primary-datatype-list &::=& primary-datatype < ',' primary-datatype-list > \\
\\
vfields &::=&  'VARIABLE'  \\
&&             'CONSTRUCTOR' < ':' spec >  < ! vfields > \\
\\
datatypes &::=& datatype < ',' datatypes > \\
\\
zfields &::=& fields \\
&&            'VARIABLE' \\
\\
fields &::=&  field < ! 'VARIABLE' >  \\
&&            field ',' fields \\
\\
spec &::=& '-' \\
&&         datatype \\
\\
field &::=& record-label ':' spec \\
\end{grammar}

\section{Grammar for regular expressions}

\begin{grammar}
regex &::=&  '/' < regex-pattern-sequence >  '/' \\
\\
regex-pattern &::=& '[' 'REGEXCHAR' '-' 'REGEXCHAR' ']'  \\
&&                  '.' \\
&&                  'REGEXCHAR' \\
&&                  '(' regex-pattern-sequence ')' \\
&&                  regex-pattern repeat-op \\
&&                  block \\
\\
repeat-op &::=& '*' \\
&&              '+' \\
&&              '?' \\
\\
regex-pattern-sequence &::=& regex-pattern < regex-pattern-sequence > \\
\end{grammar}

\section{Grammar for expressions}

\begin{grammar}
varlist &::=& 'VARIABLE' < ',' varlist > \\
\\
location &::=& 'server' \\
&&             'client' \\
&&             'native' \\
\\
special-constant &::=& 'UINTEGER' \\
&&                     'UFLOAT' \\
&&                     'true' \\
&&                     'false' \\
&&                     'STRING' \\
&&                     'CHAR' \\
\\
atomic-expression &::=& 'VARIABLE' \\
&&                      special-constant \\
&&                      parenthesized-expression \\
\\
primary-expression &::=& atomic-expression \\
&&                       '[' < exps > ']' \\
&&                       xml-forest \\
&&                       'fun' arg-list block \\
\\
constructor-expression &::=& 'CONSTRUCTOR' < parenthesized-expression > \\
\\
parenthesized-expression &::=&  '(' binop ')' \\
&&                              '(' '.' record-label ')' \\
&&                              '(' exps ')' \\
&&                              '(' labeled-exps < ! exp > ')' \\
\\
binop &::=& '-' \\
&&          '-.' \\
&&          infix-op \\
\\
infix-op &::=&  '`' 'VARIABLE' '`' \\
&&              'SYMBOL' \\
\\
postfix-expression &::=&  primary-expression \\
&&                        block \\
&&                        'spawn' block \\
&&                        postfix-expression '.' record-label \\
&&                        postfix-expression '(' < exps > ')' \\
\\
exps &::=& exp < ',' exps >  \\
\\
unary-expression &::=& '-' unary-expression \\
&&                     '-.' unary-expression \\
&&                     postfix-expression \\
&&                     constructor-expression \\
\\
infix-expression &::=& unary-expression \\
&&                     infix-expression infix-op infix-expression \\
&&                     infix-expression '~' regex \\
\\
logical-expression &::=& infix-expression \\
&&                       logical-expression !! infix-expression \\
&&                       logical-expression '&&' infix-expression \\
\\
typed-expression &::=& logical-expression < ':' datatype > \\
\\
db-expression &::=& typed-expression \\
&&                  'insert' exp 'values' exp \\
&&                  'delete' '(' table-generator ')' < where >  \\
&&                  'update' '(' table-generator ')' < where > 'set' '(' labeled-exps ')' \\
\\
where &::=& 'where' '(' exp ')' \\
\\
xml-forest &::=& xml-tree \\
&&               xml-tree xml-forest \\
\\
xmlid &::=& 'VARIABLE' \\
\\
attr-list &::=& xmlid = '"' < attr-val > '"' < attr-list >  \\
\\
attr-val &::=& block < attr-val > \\
&&             string < attr-val >  \\
\\
xml-tree &::=& '<' 'QNAME' < attr-list > '/>' \\
&&             '<' 'QNAME' < attr-list > '>' < xml-contents-list > '</' 'QNAME' '>' \\
\\
xml-contents-list &::=& xml-contents < xml-contents-list > \\
\\
xml-contents &::=& block \\
&&                 'CDATA' \\
&&                 xml-tree \\
\\
conditional-expression &::=& db-expression \\
&&                           'if' '(' exp ')' exp 'else' exp \\
\\
cases &::=& 'case' pattern '->' block-contents < cases >  \\
\\
case-expression &::=&  conditional-expression \\
&&                     'switch' '(' exp ')' '{' < cases > '}' \\
&&                     'receive' '{' < cases > '}' \\
\\
iteration-expression &::=& case-expression \\
&&                         'for' '(' generator ')' < where > < 'orderby' '(' exp ')' > exp \\
\\
generator &::=&  list-generator \\
&&               table-generator \\
\\
list-generator  &::=& 'var' pattern '<-' exp \\
table-generator &::=& 'var' pattern '<--' exp \\
\\
escape-expression &::=& iteration-expression \\
&&                      'escape' 'VARIABLE' 'in' postfix-expression \\
\\
handle-with-expression &::=& escape-expression \\
&&                          'handle' exp 'with' 'VARIABLE' '->' exp \\
\\
table-expression &::=&  handle-with-expression \\
&&                      'table' exp 'with' datatype 'from' exp \\
\\
arg-list &::=& parenthesized-pattern < arg-list >  \\
\\
binding &::=& 'var' pattern '=' exp ';' \\
&&            exp ';' \\
&&            'fun' 'VARIABLE' arg-list block \\
\\
bindings &::=& < bindings > binding \\
\\
block &::=& '{' block-contents '}' \\
\\
block-contents &::=& < bindings > < exp > < ';' >  \\
\\
labeled-exps &::=& record-label '=' exp < ',' labeled-exps > \\
\\
exp &::=& table-expression \\
&&        'database' atomic-expression < atomic-expression > < atomic-expression >  \\
\end{grammar} 

\section{Grammar for top-level declarations}

\begin{grammar}
toplevel &::=& exp ';' \\
&&             fixity 'UINTEGER' infix-op ';' \\
&&             < signature > toplevel-binding \\
\\
toplevel-binding &::=& 'var' pattern perhaps-location '=' exp ';' \\
&&                     'fun' 'VARIABLE' arg-list < location > block < ';' > \\
&&                     'fun' pattern infix-op pattern < location > block < ';' > \\
\\
fixity &::=& 'infix' \\
&&           'infixl' \\
&&           'infixr' \\
\end{grammar} 
 
\section{Terminals} 
 
\MakeShortVerb{\'} 
 
The meaning of non-literal terminals, which occur in uppercase in the 
grammar, is as follows: 
 
Identifier character an uppercase or lowercase letter, a digit, or 
underscore (\texttt{\_}).  Then a 'CONSTRUCTOR' is a non-empty sequence of 
identifiers starting with an uppercase letter and a 'VARIABLE' is a 
non-empty sequence of identifiers starting with a lowercase letter.  A 
'UINTEGER' is either '0' or a non-empty sequence of digits starting with 
a non-zero digit.  A 'SYMBOL' is a non-empty sequence of non-identifier 
characters (e.g. '++' or \texttt{\$@}).  A 'REGEXCHAR' is a character not otherwise 
employed in the grammar for regular expressions.  A 'QNAME' is a 
non-empty sequence of identifier characters and colons; the first 
character may not be a colon.  A 'UFLOAT' consists of a 'UINTEGER', a dot 
('.'), a possibly-empty sequence of digits and an optional exponent part 
consisting of the letter 'e', an optional minus ('-'), and a 'UINTEGER'. 
 
An escape character is a backslash ('\') followed by three octal 
digits, or by an uppercase or lowercase 'x' followed by two 
hexadecimal digits.  A 'STRING' consists of a possibly-empty sequence of 
characters (other than '"'), escape characters and the sequence '\"' 
between double quotes ('"').  A 'CHAR' consists of a character other than 
an escape character or the character \verb$'$ between single quotes (\verb$'$). 
'CDATA' is a non-empty sequence of characters other than '{','}','<' or '&'. 
 
\DeleteShortVerb{\'} 
 
 
\end{document} 
